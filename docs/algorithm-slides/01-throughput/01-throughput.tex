\pdfminorversion=5
\pdfobjcompresslevel=2
\documentclass[xcolor=svgnames,hyperref]{beamer}

\usepackage[T1]{fontenc}
\usepackage[utf8]{inputenc}
\usepackage[english]{babel}
\usepackage{calc}
\usepackage{graphicx}
\usepackage{textcomp}
\usepackage{tikz}
\usepackage{pgfplots}
\usepackage{hyphenat}

\usetikzlibrary{backgrounds,calc,chains,fit,positioning,shadows,shapes.geometric,shapes.multipart}

\makeatletter
\newcommand{\umlact@manual}{false}
\pgfkeys{
  /umlact/.cd,
  node/.store in=\umlact@node,
  users/.store in=\umlact@users,
  time limit/.store in=\umlact@tl,
  manual/.store in=\umlact@manual,
  type/.is choice,
  type/graph/.code={\gdef\umlact@stereotype{gpc}},
  type/activity/.code={\gdef\umlact@stereotype{pc}},
}
\newenvironment{umlactivity}[1][]{
  \begin{tikzpicture}[
    graphicnode/.style={
      draw,fill=black,color=black},
    initial/.style={
      graphicnode,shape=circle,inner sep=.2cm},
    final/.style={
      graphicnode,shape=circle,ultra thick,inner sep=.2cm},
    decision/.style={
      graphicnode,inner sep=.2cm,fill=white},
    fork/.style={
      graphicnode,inner sep=0,minimum width=0.75em,minimum height=2em},
    activity/.style={
      graphicnode,inner sep=0.8em,fill=white,rounded corners,text centered,
      text width=4em
    },
    cflow/.style={->,color=gray,very thick},
    condition/.style={auto,midway,color=black},
    alink/.style={dashed,very thick},
    #1
  ]
  \newcommand{\final}[2][]{
    \node[final,fill=white,##1] (##2) {};
    \node[final,inner sep=.125cm,##1,left=0 of ##2.center,anchor=center] (i##2) {};
  }
  \newcommand{\decision}[2][]{
    \node[decision,##1,rotate=45] (##2) {};
  }
  \newcommand{\initial}[2][]{
    \node[initial,##1] (##2) {};
  }
  \newcommand{\fork}[2][]{
    \node[fork,##1] (##2) {};
  }
  \let\join=\fork
  \newcommand{\activity}[3][]{
    \node[activity,##1] (##2) {##3};
  }
  \newcommand{\perfannotation}[2][]{
    \begingroup
    \pgfkeys{/umlact/.cd,##2}
    \node[draw,##1,fill=white,rectangle split,
      rectangle split parts=4,
      rectangle split draw splits=false,
      rectangle split part align={center,left}]
     (\umlact@node){
       �\umlact@stereotype�
       \nodepart{second}
       concurrentUsers = \umlact@users
       \nodepart{third}
       timeLimit = \umlact@tl
       \nodepart{fourth}
       manual = \umlact@manual
     };
    \endgroup
  }
}{\end{tikzpicture}}

\newcommand{\activitynode}[6][]{
  \node[draw,rounded corners,rectangle split,rectangle split parts=3,#1] (#2)
    {#3
      \nodepart{second}
      \parbox{4em}{\centering m = #4s \\ w = #5}
      \nodepart{third}
      #6}
}
\newcommand{\activitynodemm}[7][]{
  \node[draw,rounded corners,rectangle split,rectangle split parts=3,#1] (#2)
    {#3
      \nodepart{second}
      \parbox{4em}{\centering m = #4 \\ w = #5 \\ M\textsubscript{m} = #6}
      \nodepart{third}
      \makebox[0pt][l]{\phantom{?}}
      #7}
}
\tikzstyle{cflow}=[thick,->]

\newcommand*{\taggedvalues}[3][7em]{\parbox{#1}{\centering\scriptsize\guillemotleft{}#2\guillemotright{} \{#3\}}\vspace{.25em}}
\newcommand*{\talltaggedvalues}[3][7em]{\parbox{#1}{\centering\scriptsize\guillemotleft{}#2\guillemotright{}\\{}\{#3\}}\vspace{.25em}}
\newcommand*{\taggedvalue}[2]{\nohyphens{#1} = #2}
\newcommand*{\gathroughput}[2][gastep]{\taggedvalues[6.75em]{#1}{\taggedvalue{throughput}{#2}}}
\newcommand*{\gaprob}[2][gastep]{\taggedvalues[5em]{#1}{\taggedvalue{prob}{#2}}}
\newcommand*{\garespt}[2][gascenario]{\talltaggedvalues[8em]{#1}{\taggedvalue{respT}{#2}}}
\newcommand*{\gahostdem}[2][gastep]{\taggedvalues[8em]{#1}{\taggedvalue{hostDemand}{#2}}}
\newcommand*{\gactx}[2][gaanalysiscontext]{\taggedvalues[8.5em]{#1}{\taggedvalue{contextParams}{#2}}}

\newcommand*{\localann}[1]{\parbox{8.5em}{\centering\scriptsize#1}\vspace{.25em}}
\newcommand*{\slacks}[1]{\parbox{8.5em}{\centering\scriptsize#1}\vspace{.25em}}

\newcommand<>{\gatmpthroughput}[2][]{
  \temporal#3{\gathroughput#1{?}}{\color{red}\gathroughput#1{#2}}{\gathroughput#1{#2}}
}
\newcommand<>{\gatmphostdem}[2][]{
  \temporal#3{\gahostdem#1{?}}{\color{red}\gahostdem#1{#2}}{\gahostdem#1{#2}}
}

\makeatother

%%% Local Variables:
%%% mode: latex
%%% TeX-master: "qsic-agd"
%%% End:


\mode<presentation>{
  \useoutertheme[subsection=false,footline=authorinstitutetitle]{miniframes}
  \defbeamertemplate*{footline}{minipages theme}
  {
    \begin{beamercolorbox}[colsep=1.5pt]{upper separation line foot}
    \end{beamercolorbox}
    \begin{beamercolorbox}[ht=2.5ex,dp=1.125ex,%
      leftskip=.3cm,rightskip=.3cm plus1fil]{author in head/foot}%
      \leavevmode{\usebeamerfont{author in head/foot}\insertshortauthor}%
      \hfill%
      {\usebeamerfont{institute in head/foot}\usebeamercolor[fg]{institute in head/foot}\insertshortinstitute}%
    \end{beamercolorbox}%
    \begin{beamercolorbox}[ht=2.5ex,dp=1.125ex,%
      leftskip=.3cm,rightskip=.3cm plus1fil]{title in head/foot}%
      {\usebeamerfont{title in head/foot}\insertshorttitle}%
      \hfill%
      \insertframenumber{} / \inserttotalframenumber\hspace*{2ex}%
    \end{beamercolorbox}%
    \begin{beamercolorbox}[colsep=1.5pt]{lower separation line foot}
    \end{beamercolorbox}
  }

  \usecolortheme{whale}
  \usecolortheme{orchid}
  \useinnertheme{rounded}
  \setbeamertemplate{navigation symbols}{}
  \setbeamercovered{dynamic}
}
\title{SODM+T throughput inference}
\institute{Aston  University}
\author{Antonio García-Domínguez}
\date{\today}

\newenvironment{autowidthdesc}[1]{%
  \begin{list}{}{\renewcommand\makelabel[1]{\structure{##1}\hfil}%
      \settowidth\labelwidth{\makelabel{#1}}%
      \setlength\leftmargin{\labelwidth+\labelsep}}}%
  {\end{list}}

\newcommand*{\twitter}[1]{\texttt{@#1}}
\newcommand*{\evalue}[1]{e\textsuperscript{3}value\texttrademark{}\xspace}
\newcommand*{\fichero}[1]{\texttt{#1}}
\newcommand*{\ingles}[1]{\foreignlanguage{english}{\textit{#1}}}
\newcommand*{\plugin}[1]{\textit{#1}}

\renewcommand{\emph}[1]{\structure{#1}}
\newcommand*{\email}[1]{\href{mailto:#1}{#1}}

\newcommand<>{\highlight}[1]{\alt#2{{\color{red}#1}}{\color{black}#1}}
\newcommand<>{\timelimit}[1]{%
  % We need to always take up the same vertical space
  \makebox[0pt][l]{\phantom{?}}%
  \temporal#2{t = ?}{\color{red}t = #1s}{t = #1s}%
}
\newcommand<>{\appearin}[1]{\temporal#2{}{\color{red}#1}{#1}}
\newcommand<>{\computein}[1]{\temporal#2{?}{{\color{red}#1}}{#1}}

\pgfplotsset{compat=1.3}

\begin{document}

\section{Inference algorithms}

\begin{frame}{Running example: travel search engine}
  \begin{center}
    % -*- mode: latex; TeX-master: "qsic-agd" -*-

%\usetikzlibrary{backgrounds,calc,chains,fit,positioning,shadows,shapes.geometric,shapes.multipart}

\makeatletter
\newcommand{\umlact@manual}{false}
\pgfkeys{
  /umlact/.cd,
  node/.store in=\umlact@node,
  users/.store in=\umlact@users,
  time limit/.store in=\umlact@tl,
  manual/.store in=\umlact@manual,
  type/.is choice,
  type/graph/.code={\gdef\umlact@stereotype{gpc}},
  type/activity/.code={\gdef\umlact@stereotype{pc}},
}
\newenvironment{umlactivity}[1][]{
  \begin{tikzpicture}[
    graphicnode/.style={
      draw,fill=black,color=black},
    initial/.style={
      graphicnode,shape=circle,inner sep=.2cm},
    final/.style={
      graphicnode,shape=circle,ultra thick,inner sep=.2cm},
    decision/.style={
      graphicnode,inner sep=.2cm,fill=white},
    fork/.style={
      graphicnode,inner sep=0,minimum width=0.75em,minimum height=2em},
    activity/.style={
      graphicnode,inner sep=0.8em,fill=white,rounded corners,text centered,
      text width=4em
    },
    cflow/.style={->,color=gray,very thick},
    condition/.style={auto,midway,color=black},
    alink/.style={dashed,very thick},
    #1
  ]
  \newcommand{\final}[2][]{
    \node[final,fill=white,##1] (##2) {};
    \node[final,inner sep=.125cm,##1,left=0 of ##2.center,anchor=center] (i##2) {};
  }
  \newcommand{\decision}[2][]{
    \node[decision,##1,rotate=45] (##2) {};
  }
  \newcommand{\initial}[2][]{
    \node[initial,##1] (##2) {};
  }
  \newcommand{\fork}[2][]{
    \node[fork,##1] (##2) {};
  }
  \let\join=\fork
  \newcommand{\activity}[3][]{
    \node[activity,##1] (##2) {##3};
  }
  \newcommand{\perfannotation}[2][]{
    \begingroup
    \pgfkeys{/umlact/.cd,##2}
    \node[draw,##1,fill=white,rectangle split,
      rectangle split parts=4,
      rectangle split draw splits=false,
      rectangle split part align={center,left}]
     (\umlact@node){
       �\umlact@stereotype�
       \nodepart{second}
       concurrentUsers = \umlact@users
       \nodepart{third}
       timeLimit = \umlact@tl
       \nodepart{fourth}
       manual = \umlact@manual
     };
    \endgroup
  }
}{\end{tikzpicture}}

\newcommand{\activitynode}[6][]{
  \node[draw,rounded corners,rectangle split,rectangle split parts=3,#1] (#2)
    {#3
      \nodepart{second}
      \parbox{4em}{\centering m = #4s \\ w = #5}
      \nodepart{third}
      #6}
}
\newcommand{\activitynodemm}[7][]{
  \node[draw,rounded corners,rectangle split,rectangle split parts=3,#1] (#2)
    {#3
      \nodepart{second}
      \parbox{4em}{\centering m = #4 \\ w = #5 \\ M\textsubscript{m} = #6}
      \nodepart{third}
      \makebox[0pt][l]{\phantom{?}}
      #7}
}
\tikzstyle{cflow}=[thick,->]

\newcommand*{\taggedvalues}[3][7em]{\parbox{#1}{\centering\scriptsize\guillemotleft{}#2\guillemotright{} \{#3\}}\vspace{.25em}}
\newcommand*{\talltaggedvalues}[3][7em]{\parbox{#1}{\centering\scriptsize\guillemotleft{}#2\guillemotright{}\\{}\{#3\}}\vspace{.25em}}
\newcommand*{\taggedvalue}[2]{\nohyphens{#1} = #2}
\newcommand*{\gathroughput}[2][gastep]{\taggedvalues[6.75em]{#1}{\taggedvalue{throughput}{#2}}}
\newcommand*{\gaprob}[2][gastep]{\taggedvalues[5em]{#1}{\taggedvalue{prob}{#2}}}
\newcommand*{\garespt}[2][gascenario]{\talltaggedvalues[8em]{#1}{\taggedvalue{respT}{#2}}}
\newcommand*{\gahostdem}[2][gastep]{\taggedvalues[8em]{#1}{\taggedvalue{hostDemand}{#2}}}
\newcommand*{\gactx}[2][gaanalysiscontext]{\taggedvalues[8.5em]{#1}{\taggedvalue{contextParams}{#2}}}

\newcommand*{\localann}[1]{\parbox{8.5em}{\centering\scriptsize#1}\vspace{.25em}}
\newcommand*{\slacks}[1]{\parbox{8.5em}{\centering\scriptsize#1}\vspace{.25em}}

\newcommand<>{\gatmpthroughput}[2][]{
  \temporal#3{\gathroughput#1{?}}{\color{red}\gathroughput#1{#2}}{\gathroughput#1{#2}}
}
\newcommand<>{\gatmphostdem}[2][]{
  \temporal#3{\gahostdem#1{?}}{\color{red}\gahostdem#1{#2}}{\gahostdem#1{#2}}
}

\makeatother

%%% Local Variables:
%%% mode: latex
%%% TeX-master: "qsic-agd"
%%% End:


\begin{umlactivity}[node distance=5em]
  % Activity diagram nodes
  \initial{ini}
  \activity[below=1em of ini]{rq}{Receive Query}

  \fork[right=1em of rq]{forkairline}
  \activity[above right=.5em and 2.5em of forkairline.center]{qaa}{Query Airline A}
  \activity[below right=.5em and 2.5em of forkairline.center]{qab}{Query Airline B}
  \join[right=10.25em of forkairline.center]{joinairline}

  \node[draw,below=8em of forkairline,inner sep=.5em,anchor=center,rotate=-45] (dechotel) {};
  \activity[above right=.5em and 2.5em of dechotel.center]{qhc}{Query Hotel C}
  \activity[below right=.5em and 2.5em of dechotel.center]{qhd}{Query Hotel D}
  \join[right=10.5em of dechotel.center]{mergehotel}

  \activity[right=1em of mergehotel]{book}{Book}
  \final[above=1em of book]{fin}

  % Original control flows %%%%%%%%%%%%%%%%%%%%%%%%%%%%%%%%%%
  \draw[cflow]
    (ini) edge (rq)
    (rq) edge (forkairline)
    (forkairline) edge (qaa) edge (qab)
    (joinairline) edge[<-] (qaa) edge[<-] (qab)
    (dechotel) edge node[above,black] {c} (qhc) edge node[below,black] {!c} (qhd)
    (mergehotel) edge[<-] (qhc) edge[<-] (qhd) edge (book)
    (book) edge (fin);
  \draw[cflow] (joinairline) -- +(0,-4.5em) -| (dechotel);

  % % Inferred and manual restrictions %%%%%%%%%%%%%%%%%%%%%%%%
  % \begin{scope}[every node/.style={%
  %     draw,inner sep=.1em,fill=white}]
  %   \node at (rq.south)   {\highlight<4>{0s, 1}};
  %   \node at (qaa.south)  {\highlight<3>{5s, 0}};
  %   \node at (qab.south)  {\highlight<4>{0s, 1}};
  %   \node at (qhc.south)  {\highlight<4>{0s, 1}};
  %   \node at (qhd.south)  {\highlight<2>{2s, 1}};
  %   \node at (book.south) {\highlight<4>{0s, 3}};
  % \end{scope}

\end{umlactivity}

  \end{center}
\end{frame}

\begin{frame}{Throughput inference}
  \begin{overprint}
    \onslide<1> We add a global requirement (marked with \emph{req}) and
    annotate decision branches with their local probabilities.

    \onslide<2> Requests flow sequentially from the initial node.

    \onslide<3> Forks clone the same request into several concurrent ones, and
    joins wait for all incoming paths.

    \onslide<4> Decisions send the request to a branch, under some probability.

    \onslide<5> Merges bring together requests from different branches.

    \onslide<6> Done: the inferred values are annotated with \emph{calc}, to
    show they have been computed automatically.
  \end{overprint}
  \begin{center}
    % -*- mode: latex; TeX-master: "qsic-agd" -*-

%\usetikzlibrary{backgrounds,calc,chains,fit,positioning,shadows,shapes.geometric,shapes.multipart}

\makeatletter
\newcommand{\umlact@manual}{false}
\pgfkeys{
  /umlact/.cd,
  node/.store in=\umlact@node,
  users/.store in=\umlact@users,
  time limit/.store in=\umlact@tl,
  manual/.store in=\umlact@manual,
  type/.is choice,
  type/graph/.code={\gdef\umlact@stereotype{gpc}},
  type/activity/.code={\gdef\umlact@stereotype{pc}},
}
\newenvironment{umlactivity}[1][]{
  \begin{tikzpicture}[
    graphicnode/.style={
      draw,fill=black,color=black},
    initial/.style={
      graphicnode,shape=circle,inner sep=.2cm},
    final/.style={
      graphicnode,shape=circle,ultra thick,inner sep=.2cm},
    decision/.style={
      graphicnode,inner sep=.2cm,fill=white},
    fork/.style={
      graphicnode,inner sep=0,minimum width=0.75em,minimum height=2em},
    activity/.style={
      graphicnode,inner sep=0.8em,fill=white,rounded corners,text centered,
      text width=4em
    },
    cflow/.style={->,color=gray,very thick},
    condition/.style={auto,midway,color=black},
    alink/.style={dashed,very thick},
    #1
  ]
  \newcommand{\final}[2][]{
    \node[final,fill=white,##1] (##2) {};
    \node[final,inner sep=.125cm,##1,left=0 of ##2.center,anchor=center] (i##2) {};
  }
  \newcommand{\decision}[2][]{
    \node[decision,##1,rotate=45] (##2) {};
  }
  \newcommand{\initial}[2][]{
    \node[initial,##1] (##2) {};
  }
  \newcommand{\fork}[2][]{
    \node[fork,##1] (##2) {};
  }
  \let\join=\fork
  \newcommand{\activity}[3][]{
    \node[activity,##1] (##2) {##3};
  }
  \newcommand{\perfannotation}[2][]{
    \begingroup
    \pgfkeys{/umlact/.cd,##2}
    \node[draw,##1,fill=white,rectangle split,
      rectangle split parts=4,
      rectangle split draw splits=false,
      rectangle split part align={center,left}]
     (\umlact@node){
       �\umlact@stereotype�
       \nodepart{second}
       concurrentUsers = \umlact@users
       \nodepart{third}
       timeLimit = \umlact@tl
       \nodepart{fourth}
       manual = \umlact@manual
     };
    \endgroup
  }
}{\end{tikzpicture}}

\newcommand{\activitynode}[6][]{
  \node[draw,rounded corners,rectangle split,rectangle split parts=3,#1] (#2)
    {#3
      \nodepart{second}
      \parbox{4em}{\centering m = #4s \\ w = #5}
      \nodepart{third}
      #6}
}
\newcommand{\activitynodemm}[7][]{
  \node[draw,rounded corners,rectangle split,rectangle split parts=3,#1] (#2)
    {#3
      \nodepart{second}
      \parbox{4em}{\centering m = #4 \\ w = #5 \\ M\textsubscript{m} = #6}
      \nodepart{third}
      \makebox[0pt][l]{\phantom{?}}
      #7}
}
\tikzstyle{cflow}=[thick,->]

\newcommand*{\taggedvalues}[3][7em]{\parbox{#1}{\centering\scriptsize\guillemotleft{}#2\guillemotright{} \{#3\}}\vspace{.25em}}
\newcommand*{\talltaggedvalues}[3][7em]{\parbox{#1}{\centering\scriptsize\guillemotleft{}#2\guillemotright{}\\{}\{#3\}}\vspace{.25em}}
\newcommand*{\taggedvalue}[2]{\nohyphens{#1} = #2}
\newcommand*{\gathroughput}[2][gastep]{\taggedvalues[6.75em]{#1}{\taggedvalue{throughput}{#2}}}
\newcommand*{\gaprob}[2][gastep]{\taggedvalues[5em]{#1}{\taggedvalue{prob}{#2}}}
\newcommand*{\garespt}[2][gascenario]{\talltaggedvalues[8em]{#1}{\taggedvalue{respT}{#2}}}
\newcommand*{\gahostdem}[2][gastep]{\taggedvalues[8em]{#1}{\taggedvalue{hostDemand}{#2}}}
\newcommand*{\gactx}[2][gaanalysiscontext]{\taggedvalues[8.5em]{#1}{\taggedvalue{contextParams}{#2}}}

\newcommand*{\localann}[1]{\parbox{8.5em}{\centering\scriptsize#1}\vspace{.25em}}
\newcommand*{\slacks}[1]{\parbox{8.5em}{\centering\scriptsize#1}\vspace{.25em}}

\newcommand<>{\gatmpthroughput}[2][]{
  \temporal#3{\gathroughput#1{?}}{\color{red}\gathroughput#1{#2}}{\gathroughput#1{#2}}
}
\newcommand<>{\gatmphostdem}[2][]{
  \temporal#3{\gahostdem#1{?}}{\color{red}\gahostdem#1{#2}}{\gahostdem#1{#2}}
}

\makeatother

%%% Local Variables:
%%% mode: latex
%%% TeX-master: "qsic-agd"
%%% End:


\begin{umlactivity}[node distance=5em,activity/.append style={text width=6.75em,inner sep=.2em}]
  % Activity diagram nodes
  \initial{ini}
  \activity[below=1em of ini]{rq}{\gatmpthroughput<2>{10} RQ}

  \fork[right=1em of rq]{forkairline}
  \activity[above right=.25em and 1em of forkairline.center]{qaa}{\gatmpthroughput<3>{10} QAA}
  \activity[below right=.25em and 1em of forkairline.center]{qab}{\gatmpthroughput<3>{10} QAB}
  \join[right=8.75em of forkairline.center]{joinairline}

  \node[draw,below=6em of rq,inner sep=.5em,anchor=center,rotate=-45] (dechotel) {};
  \activity[above right=.25em and 6em of dechotel.center]{qhc}{\gatmpthroughput<4>{8} QHC}
  \activity[below right=.25em and 6em of dechotel.center]{qhd}{\gatmpthroughput<4>{2} QHD}
  \join[right=14em of dechotel.center]{mergehotel}

  \activity[right=1em of mergehotel]{book}{\gatmpthroughput<5>{10} B}
  \final[above=1em of book]{fin}

  % Global constraint
  \alt<1>{\tikzstyle{global}=[color=red]}{\tikzstyle{global}=[]}
  \node[draw,right=3em of joinairline,global] {%
    \highlight<1>{\gathroughput[gascenario]{(10Hz, req)}}};

  % Original control flows %%%%%%%%%%%%%%%%%%%%%%%%%%%%%%%%%%
  \draw[cflow]
    (ini) edge (rq)
    (rq) edge (forkairline)
    (mergehotel) edge (book)
    (book) edge (fin);

  \draw[cflow] (forkairline) |- (qaa);
  \draw[cflow] (forkairline) |- (qab);
  \draw[cflow] (qaa) -| (joinairline);
  \draw[cflow] (qab) -| (joinairline);

  \draw[cflow]
    (dechotel) edge node[above,black,align=center] {\highlight<1>{\gaprob{0.8}} \\ c}
    (qhc) edge node[below,black,align=center] {\highlight<1>{\gaprob{0.2}} \\ !c} (qhd);

  \draw[cflow] (qhc) -| (mergehotel);
  \draw[cflow] (qhd) -| (mergehotel);
  \draw[cflow] (joinairline) -| +(1em,-3.85em) -| (dechotel);

\end{umlactivity}

  \end{center}
\end{frame}

\end{document}
