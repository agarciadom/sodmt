\pdfminorversion=5
\pdfobjcompresslevel=2
\documentclass[xcolor=svgnames,hyperref]{beamer}

\usepackage[T1]{fontenc}
\usepackage[utf8]{inputenc}
\usepackage[english]{babel}
\usepackage{calc}
\usepackage{graphicx}
\usepackage{textcomp}
\usepackage{tikz}
\usepackage{pgfplots}
\usepackage{hyphenat}

\usetikzlibrary{backgrounds,calc,chains,fit,positioning,shadows,shapes.geometric,shapes.multipart}

\makeatletter
\newcommand{\umlact@manual}{false}
\pgfkeys{
  /umlact/.cd,
  node/.store in=\umlact@node,
  users/.store in=\umlact@users,
  time limit/.store in=\umlact@tl,
  manual/.store in=\umlact@manual,
  type/.is choice,
  type/graph/.code={\gdef\umlact@stereotype{gpc}},
  type/activity/.code={\gdef\umlact@stereotype{pc}},
}
\newenvironment{umlactivity}[1][]{
  \begin{tikzpicture}[
    graphicnode/.style={
      draw,fill=black,color=black},
    initial/.style={
      graphicnode,shape=circle,inner sep=.2cm},
    final/.style={
      graphicnode,shape=circle,ultra thick,inner sep=.2cm},
    decision/.style={
      graphicnode,inner sep=.2cm,fill=white},
    fork/.style={
      graphicnode,inner sep=0,minimum width=0.75em,minimum height=2em},
    activity/.style={
      graphicnode,inner sep=0.8em,fill=white,rounded corners,text centered,
      text width=4em
    },
    cflow/.style={->,color=gray,very thick},
    condition/.style={auto,midway,color=black},
    alink/.style={dashed,very thick},
    #1
  ]
  \newcommand{\final}[2][]{
    \node[final,fill=white,##1] (##2) {};
    \node[final,inner sep=.125cm,##1,left=0 of ##2.center,anchor=center] (i##2) {};
  }
  \newcommand{\decision}[2][]{
    \node[decision,##1,rotate=45] (##2) {};
  }
  \newcommand{\initial}[2][]{
    \node[initial,##1] (##2) {};
  }
  \newcommand{\fork}[2][]{
    \node[fork,##1] (##2) {};
  }
  \let\join=\fork
  \newcommand{\activity}[3][]{
    \node[activity,##1] (##2) {##3};
  }
  \newcommand{\perfannotation}[2][]{
    \begingroup
    \pgfkeys{/umlact/.cd,##2}
    \node[draw,##1,fill=white,rectangle split,
      rectangle split parts=4,
      rectangle split draw splits=false,
      rectangle split part align={center,left}]
     (\umlact@node){
       �\umlact@stereotype�
       \nodepart{second}
       concurrentUsers = \umlact@users
       \nodepart{third}
       timeLimit = \umlact@tl
       \nodepart{fourth}
       manual = \umlact@manual
     };
    \endgroup
  }
}{\end{tikzpicture}}

\newcommand{\activitynode}[6][]{
  \node[draw,rounded corners,rectangle split,rectangle split parts=3,#1] (#2)
    {#3
      \nodepart{second}
      \parbox{4em}{\centering m = #4s \\ w = #5}
      \nodepart{third}
      #6}
}
\newcommand{\activitynodemm}[7][]{
  \node[draw,rounded corners,rectangle split,rectangle split parts=3,#1] (#2)
    {#3
      \nodepart{second}
      \parbox{4em}{\centering m = #4 \\ w = #5 \\ M\textsubscript{m} = #6}
      \nodepart{third}
      \makebox[0pt][l]{\phantom{?}}
      #7}
}
\tikzstyle{cflow}=[thick,->]

\newcommand*{\taggedvalues}[3][7em]{\parbox{#1}{\centering\scriptsize\guillemotleft{}#2\guillemotright{} \{#3\}}\vspace{.25em}}
\newcommand*{\talltaggedvalues}[3][7em]{\parbox{#1}{\centering\scriptsize\guillemotleft{}#2\guillemotright{}\\{}\{#3\}}\vspace{.25em}}
\newcommand*{\taggedvalue}[2]{\nohyphens{#1} = #2}
\newcommand*{\gathroughput}[2][gastep]{\taggedvalues[6.75em]{#1}{\taggedvalue{throughput}{#2}}}
\newcommand*{\gaprob}[2][gastep]{\taggedvalues[5em]{#1}{\taggedvalue{prob}{#2}}}
\newcommand*{\garespt}[2][gascenario]{\talltaggedvalues[8em]{#1}{\taggedvalue{respT}{#2}}}
\newcommand*{\gahostdem}[2][gastep]{\taggedvalues[8em]{#1}{\taggedvalue{hostDemand}{#2}}}
\newcommand*{\gactx}[2][gaanalysiscontext]{\taggedvalues[8.5em]{#1}{\taggedvalue{contextParams}{#2}}}

\newcommand*{\localann}[1]{\parbox{8em}{\centering\scriptsize#1}\vspace{.25em}}
\newcommand*{\slacks}[1]{\parbox{8em}{\centering\scriptsize#1}\vspace{.25em}}

\newcommand*{\throughput}[1]{\parbox{6.75em}{\centering\scriptsize throughput = #1}}
\newcommand*{\prob}[1]{\parbox{5em}{\centering\scriptsize\{prob = #1\}}}
\newcommand<>{\tmpthroughput}[2][]{\temporal#3{\throughput#1{?}}{\color{red}\throughput#1{#2}}{\throughput#1{#2}}}

\newcommand<>{\gatmpthroughput}[2][]{
  \temporal#3{\gathroughput#1{?}}{\color{red}\gathroughput#1{#2}}{\gathroughput#1{#2}}
}
\newcommand<>{\gatmphostdem}[2][]{
  \temporal#3{\gahostdem#1{?}}{\color{red}\gahostdem#1{#2}}{\gahostdem#1{#2}}
}

\makeatother

%%% Local Variables:
%%% mode: latex
%%% TeX-master: "qsic-agd"
%%% End:


\mode<presentation>{
  \usecolortheme{whale}
  \usecolortheme{orchid}
  \useinnertheme{rounded}
  \setbeamertemplate{navigation symbols}{}
  \setbeamercovered{dynamic}
}
\title{SODM+T Incremental Time Limit Inference}
\institute{Aston University}
\author{Antonio García-Domínguez}
\date{\today}

\newenvironment{autowidthdesc}[1]{%
  \begin{list}{}{\renewcommand\makelabel[1]{\structure{##1}\hfil}%
      \settowidth\labelwidth{\makelabel{#1}}%
      \setlength\leftmargin{\labelwidth+\labelsep}}}%
  {\end{list}}

\newcommand*{\twitter}[1]{\texttt{@#1}}
\newcommand*{\evalue}[1]{e\textsuperscript{3}value\texttrademark{}\xspace}
\newcommand*{\fichero}[1]{\texttt{#1}}
\newcommand*{\ingles}[1]{\foreignlanguage{english}{\textit{#1}}}
\newcommand*{\plugin}[1]{\textit{#1}}

\renewcommand{\emph}[1]{\structure{#1}}
\newcommand*{\email}[1]{\href{mailto:#1}{#1}}

\newcommand<>{\highlight}[1]{\alt#2{{\color{red}#1}}{\color{black}#1}}
\newcommand<>{\timelimit}[1]{%
  % We need to always take up the same vertical space
  \makebox[0pt][l]{\phantom{?}}%
  \temporal#2{t = ?}{\color{red}t = #1s}{t = #1s}%
}
\newcommand<>{\appearin}[1]{\temporal#2{}{\color{red}#1}{#1}}
\newcommand<>{\computein}[1]{\temporal#2{?}{{\color{red}#1}}{#1}}

\pgfplotsset{compat=1.3}

\begin{document}

\section{Inference algorithms}

\begin{frame}{Incremental time limit inference}
  \begin{overprint}
    \onslide<1> All paths from the initial node to the final nodes must finish
    within 10 seconds. We will infer the resulting time limits for each activity.

    \onslide<2> Activities are annotated with $m + w S$. $m$ is the
    \emph{minimum time limit}, $w$ is the \emph{weight} and $S$ will be the
    computed \emph{slack per unit of weight}.

    \onslide<3> Most activities will have $m=0$. $w$ will be an
    estimation of their relative computational cost.

    \onslide<4> Other activities may have Service Level Agreements in
    place, with a previously agreed time limit.

    \onslide<5> Finally, some activities may combine a fixed part with the
    variable part computed by the algorithm.

    \onslide<6> We need to compute the total minimum time limit and
    weight of the strictest subpath from each activity. We start from
    the final node.

    \onslide<7> We continue in reverse topological order.

    \onslide<8> Between (0, 4) and (2, 4), (2, 4) is always stricter:
    discard (0, 4) and send (2, 4) back to the join node.

    \onslide<9> This time, we propagate (7, 4) back up. The strictest
    path in the graph from the initial node to a final node is (7, 5).

    \onslide<10> We send 10s into RQ. RQ uses up $0 + 1 (10-7)/5 =
    0.6$s and sends the remaining 9.4s to the fork node.

    \onslide<11> QAA uses up exactly 5s and sends the remaining 4.4s
    into the join node.

    \onslide<12> The join node does not use up any time.

    \onslide<13> QHD receives 4.4s, uses up $2 + 1 (4.4-2)/4 = 2.6$s
    and sends the rest to the merge node.

    \onslide<14> B receives 1.8s and uses up the remaining $0 +
    3(1.8-0)/3 = 1.8$s.

    \onslide<15> We back up into QHC. At first, QHC uses up 1.1s and
    sends 3.3s into the merge node.

    \onslide<16> However, the merge node already received 1.8s: we
    will dedicate the extra 1.5s to QHC.

    \onslide<17> We back up into QAB. Again, QAB only uses up 3.18s
    and sends too much time to the merge node.

    \onslide<18> We reuse the extra 1.82s into QAB, and we are done.
  \end{overprint}
  \begin{center}
    % -*- mode: latex; TeX-master: "qsic-agd" -*-

%\usetikzlibrary{backgrounds,calc,chains,fit,positioning,shadows,shapes.geometric,shapes.multipart}

\makeatletter
\newcommand{\umlact@manual}{false}
\pgfkeys{
  /umlact/.cd,
  node/.store in=\umlact@node,
  users/.store in=\umlact@users,
  time limit/.store in=\umlact@tl,
  manual/.store in=\umlact@manual,
  type/.is choice,
  type/graph/.code={\gdef\umlact@stereotype{gpc}},
  type/activity/.code={\gdef\umlact@stereotype{pc}},
}
\newenvironment{umlactivity}[1][]{
  \begin{tikzpicture}[
    graphicnode/.style={
      draw,fill=black,color=black},
    initial/.style={
      graphicnode,shape=circle,inner sep=.2cm},
    final/.style={
      graphicnode,shape=circle,ultra thick,inner sep=.2cm},
    decision/.style={
      graphicnode,inner sep=.2cm,fill=white},
    fork/.style={
      graphicnode,inner sep=0,minimum width=0.75em,minimum height=2em},
    activity/.style={
      graphicnode,inner sep=0.8em,fill=white,rounded corners,text centered,
      text width=4em
    },
    cflow/.style={->,color=gray,very thick},
    condition/.style={auto,midway,color=black},
    alink/.style={dashed,very thick},
    #1
  ]
  \newcommand{\final}[2][]{
    \node[final,fill=white,##1] (##2) {};
    \node[final,inner sep=.125cm,##1,left=0 of ##2.center,anchor=center] (i##2) {};
  }
  \newcommand{\decision}[2][]{
    \node[decision,##1,rotate=45] (##2) {};
  }
  \newcommand{\initial}[2][]{
    \node[initial,##1] (##2) {};
  }
  \newcommand{\fork}[2][]{
    \node[fork,##1] (##2) {};
  }
  \let\join=\fork
  \newcommand{\activity}[3][]{
    \node[activity,##1] (##2) {##3};
  }
  \newcommand{\perfannotation}[2][]{
    \begingroup
    \pgfkeys{/umlact/.cd,##2}
    \node[draw,##1,fill=white,rectangle split,
      rectangle split parts=4,
      rectangle split draw splits=false,
      rectangle split part align={center,left}]
     (\umlact@node){
       �\umlact@stereotype�
       \nodepart{second}
       concurrentUsers = \umlact@users
       \nodepart{third}
       timeLimit = \umlact@tl
       \nodepart{fourth}
       manual = \umlact@manual
     };
    \endgroup
  }
}{\end{tikzpicture}}

\newcommand{\activitynode}[6][]{
  \node[draw,rounded corners,rectangle split,rectangle split parts=3,#1] (#2)
    {#3
      \nodepart{second}
      \parbox{4em}{\centering m = #4s \\ w = #5}
      \nodepart{third}
      #6}
}
\newcommand{\activitynodemm}[7][]{
  \node[draw,rounded corners,rectangle split,rectangle split parts=3,#1] (#2)
    {#3
      \nodepart{second}
      \parbox{4em}{\centering m = #4 \\ w = #5 \\ M\textsubscript{m} = #6}
      \nodepart{third}
      \makebox[0pt][l]{\phantom{?}}
      #7}
}
\tikzstyle{cflow}=[thick,->]

\newcommand*{\taggedvalues}[3][7em]{\parbox{#1}{\centering\scriptsize\guillemotleft{}#2\guillemotright{} \{#3\}}\vspace{.25em}}
\newcommand*{\talltaggedvalues}[3][7em]{\parbox{#1}{\centering\scriptsize\guillemotleft{}#2\guillemotright{}\\{}\{#3\}}\vspace{.25em}}
\newcommand*{\taggedvalue}[2]{\nohyphens{#1} = #2}
\newcommand*{\gathroughput}[2][gastep]{\taggedvalues[6.75em]{#1}{\taggedvalue{throughput}{#2}}}
\newcommand*{\gaprob}[2][gastep]{\taggedvalues[5em]{#1}{\taggedvalue{prob}{#2}}}
\newcommand*{\garespt}[2][gascenario]{\talltaggedvalues[8em]{#1}{\taggedvalue{respT}{#2}}}
\newcommand*{\gahostdem}[2][gastep]{\taggedvalues[8em]{#1}{\taggedvalue{hostDemand}{#2}}}
\newcommand*{\gactx}[2][gaanalysiscontext]{\taggedvalues[8.5em]{#1}{\taggedvalue{contextParams}{#2}}}

\newcommand*{\localann}[1]{\parbox{8em}{\centering\scriptsize#1}\vspace{.25em}}
\newcommand*{\slacks}[1]{\parbox{8em}{\centering\scriptsize#1}\vspace{.25em}}

\newcommand*{\throughput}[1]{\parbox{6.75em}{\centering\scriptsize throughput = #1}}
\newcommand*{\prob}[1]{\parbox{5em}{\centering\scriptsize\{prob = #1\}}}
\newcommand<>{\tmpthroughput}[2][]{\temporal#3{\throughput#1{?}}{\color{red}\throughput#1{#2}}{\throughput#1{#2}}}

\newcommand<>{\gatmpthroughput}[2][]{
  \temporal#3{\gathroughput#1{?}}{\color{red}\gathroughput#1{#2}}{\gathroughput#1{#2}}
}
\newcommand<>{\gatmphostdem}[2][]{
  \temporal#3{\gahostdem#1{?}}{\color{red}\gahostdem#1{#2}}{\gahostdem#1{#2}}
}

\makeatother

%%% Local Variables:
%%% mode: latex
%%% TeX-master: "qsic-agd"
%%% End:


\begin{umlactivity}[%
  node distance=5em,%
  activity/.append style={text width=8em,inner sep=.2em},%
  senttime/.style={draw,color=black,font=\scriptsize,fill=white,inner sep=.2em},%
]
  % Activity diagram nodes
  \initial{ini}
  \activity[below=.5em of ini]{rq}{\localann{\highlight<2-3>{(sRQ)}} RQ}

  \fork[right=.5em of rq]{forkairline}
  \activity[above right=.25em and .75em of forkairline.center]{qaa}{%
    \localann{\highlight<2,4>{(5)}} QAA}
  \activity[below right=.25em and .75em of forkairline.center]{qab}{%
    \localann{\highlight<2-3>{(sAB, rep=3)}} QAB}
  \join[right=9.5em of forkairline.center]{joinairline}

  \node[draw,below=7.25em of forkairline,inner sep=.5em,anchor=center,rotate=-45] (dechotel) {};
  \activity[above right=.25em and .75em of dechotel.center]{qhc}{%
    \localann{\highlight<2-3>{(sHC)}} QHC}
  \activity[below right=.25em and .75em of dechotel.center]{qhd}{%
    \localann{\highlight<2,5>{(2 + sHD)}} QHD}
  \join[right=9.5em of dechotel.center]{mergehotel}

  \activity[right=.5em of mergehotel]{book}%
  {\localann{\highlight<2-3>{(3 $\times$ sB)}} B}
  \final[above=.5em of book]{fin}

  % Global constraint
  \alt<1-2>{\tikzstyle{global}=[color=red]}{\tikzstyle{global}=[]}
  \node[draw,right=1em of joinairline,global,align=center] {%
    \highlight<1>{\scriptsize time limit = 10s}
    \\[.5em]{}
    \temporal<2>{%
      \slacks{sRQ = ?, sAB = ?,\\{}sHC = ?, sHD = ?,\\{}sB = ?}}{%
      \color{red}
      \slacks{sRQ = ?, sAB = ?,\\{}sHC = ?, sHD = ?,\\{}sB = ?}}{%
      \slacks{%
        sRQ = \computein<10>{0.6},
        sAB = \computein<17-18>{\alt<17>{1.06}{1.67}},
        \\{}sHC = \computein<15-16>{\alt<15>{1.1}{2.6}},
        sHD = \computein<13>{0.6},
        \\{}sB = \computein<14>{0.6}}}};

  % Original control flows %%%%%%%%%%%%%%%%%%%%%%%%%%%%%%%%%%
  \draw[cflow]
    (ini) edge node (tinirq) {} (rq)
    (rq) edge node (trqforkairline) {} (forkairline)
    (mergehotel) edge node (tmergehotelbook) {} (book)
    (book) edge node (tbookfin) {} (fin);

  \draw[cflow] (dechotel) |- node (tdechotelqhc) {} (qhc);
  \draw[cflow] (dechotel) |- node (tdechotelqhd) {} (qhd);
  \draw[cflow] (forkairline) |- node (tforkairlineqaa) {} (qaa);
  \draw[cflow] (forkairline) |- node (tforkairlineqab) {} (qab);
  \draw[cflow] (qaa) -| node (tqaajoinairline) {} (joinairline);
  \draw[cflow] (qab) -| node (tqabjoinairline) {} (joinairline);
  \draw[cflow] (qhc) -| node (tqhcmergehotel) {} (mergehotel);
  \draw[cflow] (qhd) -| node (tqhdmergehotel) {} (mergehotel);
  \draw[cflow] (joinairline)
    -| +(1em,-4.125em)
    -| ($(-1.5em,0)+(dechotel)$)
       node (tjoinairlinedechotel) {}
    -- (dechotel);

  % Algorithm traces %%%%%%%%%%%%%%%%%%%%%%%%%%%%%%%%%%%%%%%

  % Time sent into each node
  \begin{scope}[every node/.style={senttime}]
    \node<10-> at (tinirq) {\highlight<10>{10s}};
    \node<10-> at (trqforkairline) {\highlight<10>{9.4s}};
    \node<11-> at (tforkairlineqaa) {\highlight<11>{9.4s}};
    \node<17-> at (tforkairlineqab) {\highlight<17>{9.4s}};
    \node<11-> at (tqaajoinairline) {\highlight<11>{4.4s}};
    \node<17-> at (tqabjoinairline) {%
      \temporal<18>{{\color{red}6.22s}}{{\color{red}4.4s}}{4.4s}};
    \node<12-> at (tjoinairlinedechotel) {\highlight<12>{4.4s}};
    \node<15-> at (tdechotelqhc) {\highlight<15>{4.4s}};
    \node<13-> at (tdechotelqhd) {\highlight<13>{4.4s}};
    \node<15-> at (tqhcmergehotel) {%
      \temporal<16>{{\color{red}3.3s}}{{\color{red}1.8s}}{1.8s}};
    \node<13-> at (tqhdmergehotel) {\highlight<13>{1.8s}};
    \node<14-> at (tmergehotelbook) {\highlight<14>{1.8s}};
    \node<14-> at (tbookfin) {\highlight<14>{0s}};
  \end{scope}

  % Aggregated path information
  \begin{scope}[%
    every node/.style={draw,rounded corners,inner sep=.15em,fill=structure!20},%
    anchor=south east,font=\small,%
  ]
    \node<9-> at (rq.south east) {\highlight<9>{7, 5}};
    \node<8-> at (qaa.south east) {\highlight<8>{7, 4}};
    \node<8-> at (qab.south east) {\highlight<8>{2, 7}};
    \node<7-> at (qhc.south east) {\highlight<7>{0, 4}};
    \node<7-> at (qhd.south east) {\highlight<7>{2, 4}};
    \node<6-> at (book.south east) {\highlight<6>{0, 3}};
  \end{scope}

\end{umlactivity}

  \end{center}
\end{frame}

\end{document}

%%% Local Variables:
%%% mode: latex
%%% TeX-master: t
%%% ispell-dictionary: american
%%% End:
